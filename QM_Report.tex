\documentclass[12pt, a4paper, titlepage]{article}
\author{陈彦熹}
\title{Q-M算法的Matlab实现\ \ 实验报告}

\usepackage{xeCJK}
\usepackage{fontspec}
\usepackage{indentfirst}
\usepackage{setspace}

\setmainfont{NSimSun}
\setlength{\parindent}{2.45em}
\linespread{1.2}

\begin{document}
	\maketitle
	\section{实验目的}
	本实验根据Q-M算法的原理,实现一个能处理十变量及以下的逻辑函数的化简工具。理论上该工具也可化简十变量以上的逻辑函数,但是无法保证运行效率。
	\section{理论依据}
	本实验的理论依据为Q-M算法,具体实现包括以下两个步骤:
	\begin{itemize}
		\item[-]搜索本原蕴含项
		\item[-]寻找最小覆盖
	\end{itemize}
	对于第一步,Q-M算法的基本思想与卡诺图一致,即找出相邻最小项,消除变量因子;对于第二步,本实验采用了Petrick's Method\footnote{Wikipedia. Petrick's Method. https://en.wikipedia.org/wiki/Petrick's\_method},以便于计算机实现。
	\section{实验结果}
	本实验以Matlab函数的形式实现逻辑化简工具。
	
	函数Q\_M接受三个参量:N表示逻辑函数变量个数,一般不超过10;m是逻辑函数规范形式的最小项代号组成的向量,元素范围从0到$2^N-1$;d是逻辑函数规范形式的无关项代号组成的向量,元素范围同m。函数实现对于输入参量的合法性检查包括m与d元素范围,当出现输入错误时给出警告信息并终止函数运行;另外函数预先对m与d进行去重复元素处理。函数没有检查m与d中是否存在相同元素,根据函数具体实现,对于这种情况,将该重复元素视为最小项而非无关项(除非m与d元素数目之和恰等于$2^N$,此时函数会错误地返回`1')。
	
	函数Q\_M返回值lgcExpr为字符串,表示经过化简得到的最简二级与或形式逻辑表达式,字母A到J为函数变量;变量取非用X'形式表示。
	\section{具体实现}
		\subsection{预处理及初始化}
		对输入参量进行检查与处理,并且定义部分后续所需变量;将最小项与无关项的代号通过函数dec2bin转化成N位二进制字串,存储在矩阵binstr中。
		\subsection{结合最小项与无关项}
		比较binstr各行字符串,若只有一位不同,则必有两串该位字符分别为0和1,两串合并,该位字符置为`-',并加入到nextbinstr中,cmb\_flg中相应值置1;一轮循环结束后,cmb\_flg为0的对应项不可再合并,则放入矩阵final,用nextbinstr更新binstr,更新cmb\_flg,进行新一轮循环。注意到这里并没有按照手工做法,先将各项按照包含1的个数进行分组,是因为这对于运行效率没有大的影响;反而是每次循环后应对binstr进行去重复处理,否则binstr中出现大量重复项,严重影响运行效率。
		\subsection{寻找最小覆盖}
		构造本原蕴含图ptk,应用Petrick's Method寻找最小覆盖。首先找出本质本原蕴含项,将该项在final中的对应项加入到rslt中,并且删除ptk中对应的行和列;假如ptk尚不为空,说明还需要寻找尽可能少的本原蕴含项来覆盖剩余的最小项。最终选择的项对应final中的项也加入到rslt中。
		\subsection{输出结果}
		将rslt中的各行字符串转化为与或形式逻辑函数式中的各项。比如对于四变量输入,rslt中一行字符串``1-00''转化为逻辑函数中的``AC'D' ''。
	\section{缺陷与不足}
	本实验实现的Q\_M函数,最大的缺陷在于寻找最小覆盖过程中,处理ptk非空情况时所使用的一个for循环(第114行)。该处理过程涉及到类似于数学中因式相乘做展开的过程: $(A+B)(A+C)=(A^2+AC+BA+BC)$。对于Q\_M函数,各因式存储在cbcell中,并且长度不一,因此不得不引入向量cbarr,模拟一个轮转循环取元素相乘的过程。假如cbcell中包含3、4、5元素的向量各1个,那么while循环需要执行$3\times 4\times 5=60$次,得到60个展开因子。由于Matlab作为一种解释性语言,循环效率较低,因此这部分代码成为了整个函数性能的瓶颈,尤其当本原蕴含图较为稠密时,这部分循环会消耗大量时间。事实上,若cbcell中的元素等长,比如cbcell包含p个二元素向量,那么可以用一个整数k从0依次加1到$2^p-1$,对每个k值,通过代码dec2bin(k,p)-`0'+1,即可得到一个用于选取各因式元素的向量cbarr,这样做的效率远远高于本实验的实现方式。
\end{document}